\chapter{迭代器}\label{ch15}

\emph{It was the end of a very long day.}

\begin{flushright}
    ——Phil
\end{flushright}

一个\emph{迭代器(iterator)}可以产生一个指的序列,通常会使用一个循环来进行处理。Rust的标准库提供了遍历vector、字符串、哈希表和其他集合的迭代器,以及从一个输入流中产生若干行文本的迭代器、到达网络服务器的连接的迭代器、通过通道从其他线程接收到的值的迭代器,等等。当然,你可以实现自己的迭代器。Rust的\texttt{for}循环提供了一种自然地使用迭代器的语法,但迭代器自身也提供了丰富的方法集合用于映射、过滤、连接、收集等用途。

Rust的迭代器灵活、表达力强、高效。考虑下面的函数,它返回前\texttt{n}个正数的和(通常也被称为\emph{第n个三角数(nth triangle number)}:
\begin{minted}{Rust}
    fn triangle(n: i32) -> i32 {
        let mut sum = 0;
        for i in 1..=n {
            sum += i;
        }
        sum
    }
\end{minted}

表达式\texttt{1..=n}是一个\texttt{RangeInclusive<i32>}值。一个\texttt{RangeInclusive<i32>}是一个产生从起点到终点的所有整数的迭代器(包含起点和终点),因此你可以将它用作\texttt{for}循环的操作数来求\texttt{1}到\texttt{n}的和。

但迭代器也有一个\texttt{fold}方法,你可以使用它实现如下的等价定义:
\begin{minted}{Rust}
    fn triangle(n: i32) -> i32 {
        (1..=n).fold(0, |sum, item| sum + item)
    }
\end{minted}

以\texttt{0}作为起始的总和,\texttt{fold}会获取\texttt{1..=n}产生的每个值,然后用总和和产生的值调用闭包\texttt{|sum, item| sum + item},每一次闭包的返回值就是新的总和。它最后返回的值就是\texttt{fold}自身返回的值——在这个例子中,就是整个序列的总和。如果你习惯使用\texttt{for}和\texttt{while}循环,那么这看起来会有些奇怪,但一旦你习惯了它,\texttt{fold}就是一个可读性强而简洁的替代方案。

这种写法是函数式编程语言的标准写法,这使得表达式有更强的表现力。但Rsut的迭代器是精心设计的,为了保证编译器可以把它们翻译成优秀的机器代码。在release构建模式下构建上面第二个定义时,Rust知道\texttt{fold}的定义,并且把它内联进\texttt{triangle}。然后,闭包\texttt{|sum, item| sum + item}也会被内联。最后,Rust会检查组合之后的代码,然后发现有一种更简单的方法计算从1到\texttt{n}的和:和总是等于\texttt{n * (n+1) / 2}。Rust会把\texttt{triangle}的整个函数体,包括循环、闭包等所有内容,变成一次乘法指令和一些其他的位运算。

这个例子恰巧可以转换成简单的算术,但在更复杂的使用中迭代器也可以表现的很好。它们是Rust提供灵活抽象的同时只有很小甚至没有开销的另一个例子。

在本章中,我们将会解释:
\begin{itemize}
    \item \texttt{Iterator}和\texttt{IntoIterator} trait,它们是Rust迭代器的基础
    \item 经典迭代器管道的三个阶段:从初始的值创建一个迭代器;通过选择或处理值将一种迭代器变成另一种;消耗迭代器产生的值
    \item 如何为自己的类型实现迭代器
\end{itemize}

迭代器有很多方法,所以一旦你了解了大概的思路,就可以跳过一节。但迭代器在Rust的习惯用法中非常普遍,熟悉这些随附的工具对掌握这门语言至关重要。

\section{\texttt{Iterator}与\texttt{IntoIterator trait}}\label{iter}

一个迭代器是任何实现了\texttt{std::iter::Iterator} trait的类型:
\begin{minted}{Rust}
    trait Iterator {
        type Item;
        fn next(&mut self) -> Option<Self::Item>;
        ... // 很多默认方法
    }
\end{minted}

\texttt{Item}是迭代器产生的值的类型。\texttt{next}方法可能返回\texttt{Some(v)},其中\texttt{v}是迭代器的下一个值;或者返回\texttt{None},表示已经到达序列的终点。这里我们省略了\texttt{Iterator}的很多默认方法;我们将在本章的剩余部分分别介绍它们。

如果有一种自然的方法从一个类型上迭代,那么这个类型可以实现\texttt{std::iter::IntoIterator},它的\texttt{into\_iter}方法获取一个值并返回一个迭代它的迭代器:
\begin{minted}{Rust}
    trait IntoIterator where Self::IntoIterator: Iterator<Item=Self::Item> {
        type Item;
        type IntoIter: Iterator;
        fn into_iter(self) -> Self::IntoIter;
    }
\end{minted}

\texttt{IntoIter}是迭代器自身的类型,\texttt{Item}是它产生的值的类型。我们称所有实现了\texttt{IntoIterator}的类型为\emph{可迭代对象(iterable)},因为你可以迭代它。

Rust的\texttt{for}循环将这些部分漂亮地组合在一起。为了迭代一个迭代器的元素,你可以写:
\begin{minted}{Rust}
    println!("There's:");
    let v = vec!["antimony", "arsenic", "alumium", "selenium"];

    for element in &v {
        println!("{}", element);
    }
\end{minted}

在底层,每一个\texttt{for}循环只是\texttt{IntoIterator}和\texttt{Iterator}的方法调用的缩写:
\begin{minted}{Rust}
    let mut iterator = (&v).into_iter();
    while let Some(element) = iterator.next() {
        println!("{}", element);
    }
\end{minted}

\texttt{for}循环使用了\texttt{IntoIterator::into\_iter}来把操作数\texttt{\&v}转换成一个迭代器,然后重复调用\texttt{Iterator::next}。每一次返回\texttt{Some(element)}时,\texttt{for}循环会执行循环体;如果它返回\texttt{None},循环会终止。

考虑这个例子,其中有一些迭代器的术语:
\begin{itemize}
    \item 正如我们所说,\emph{迭代器(iterator)}是任何实现了\texttt{Iterator}的类型。
    \item \emph{可迭代对象(iterable)}是任何实现了\texttt{IntoIterator}的类型:你可以通过调用它的\texttt{into\_iter}方法获得一个迭代它的迭代器。这个例子中vector的引用\texttt{\&v}就是可迭代对象。
    \item 一个迭代器\emph{产生(produce)}值。
    \item 迭代器产生的值是\emph{item}。这里,item是\texttt{"antimony", "arsenic},等等。
    \item 接受迭代器产生的item的代码是\emph{消费者(consumer)}。这个例子中,\texttt{for}循环就是消费者。
\end{itemize}

尽管\texttt{for}循环总是调用操作数的\texttt{into\_iter},你也可以直接向\texttt{for}循环传递迭代器;例如,当你在\texttt{Range}上循环时就是这种情况。所有的迭代器都会自动实现\texttt{IntoIterator},它们的\texttt{into\_iter}方法简单地返回迭代器自身。

如果在迭代器返回了\texttt{None}之后,你再调用它的\texttt{next}方法,那么\texttt{Iterator} trait并没有指定这种情况下该怎么做。大多数会再次返回\texttt{None},但不是所有。(如果这导致了问题,“\nameref{fuse}”中介绍的\texttt{fuse}适配器可能会有帮助。)

\section{创建迭代器}
Rust标准库文档中详细解释了每种类型提供哪些种类的迭代器,但标准库提供了一些通用的约定来帮助你找到需要的迭代器。

\subsection{\texttt{iter}和\texttt{iter\_mut}方法}
大多数集合类型提供\texttt{iter}和\texttt{iter\_mut}方法,它们返回一个迭代器,迭代器会产生每一个item的共享引用或可变引用。数组切片例如\texttt{\&[T]}和\texttt{\&mut [T]}也有\texttt{iter}和\texttt{iter\_mut}方法。除了使用\texttt{for}循环自动处理之外,这些方法是最常用的获得迭代器的方法:
\begin{minted}{Rust}
    let v = vec![4, 20, 12, 8, 6];
    let mut iterator = v.iter();
    assert_eq!(iterator.next(), Some(&4));
    assert_eq!(iterator.next(), Some(&20));
    assert_eq!(iterator.next(), Some(&12));
    assert_eq!(iterator.next(), Some(&8));
    assert_eq!(iterator.next(), Some(&6));
    assert_eq!(iterator.next(), None);
\end{minted}

这个迭代器的item类型是\texttt{\&i32}:每一次调用\texttt{next}都会产生下一个元素的引用,直到到达vector的终点。

每一个类型都可以实现\texttt{iter}和\texttt{iter\_mut},不管它们实现的方式是什么。\texttt{std::path::Path}的\texttt{iter}返回的迭代器一次产生路径的一段:
\begin{minted}{Rust}
    use std::ffi::OsStr;
    use std::path::Path;

    let path = Path::new("C:/Users/JimB/Downloads/Fedora.iso");
    let mut iterator = path.iter();
    assert_eq!(iterator.next(), Some(OsStr::new("C:")));
    assert_eq!(iterator.next(), Some(OsStr::new("Users")));
    assert_eq!(iterator.next(), Some(OsStr::new("JimB")));
    ...
\end{minted}

这个迭代器的item类型是\texttt{\&std::ffi::OsStr},它是操作系统调用接受的一种字符串类型的引用切片。

如果某个类型有不止一种迭代方式,那么这个类型通常为每种遍历方式提供特定的方法,因为这时普通的\texttt{iter}方法将会导致歧义。例如,\texttt{\&str}字符串切片类型没有\texttt{iter}方法。作为替代,假设\texttt{s}是\texttt{\&str},那么\texttt{s.bytes()}返回一个产生\texttt{s}的每个字节的迭代器,而\texttt{s.chars()}会以UTF-8编码解析它的内容,然后产生每一个Unicode字符。

\subsection{\texttt{IntoIterator}实现}
当一个类型实现了\texttt{IntoIterator}之后,你可以自己调用它的\texttt{into\_iter}方法,正如\texttt{for}循环做的一样:
\begin{minted}{Rust}
    // 你通常应该使用HashSet,但它的迭代顺序是不确定的,
    // 因此这个例子中BTreeSet会工作得更好。
    use std::collections::BTreeSet;
    let mut favorites = BTreeSet::new();
    favorites.insert("Lucy in the Sky With Diamonds".to_string());
    favorites.insert("Liebesträume No. 3".to_string());

    let mut it = favorites.into_iter();
    assert_eq!(it.next(), Some("Liebesträume No. 3".to_string()));
    assert_eq!(it.next(), Some("Lucy in the Sky With Diamonds".to_string()));
    assert_eq!(it.next(), None);
\end{minted}

大多数集合实际上都提供了好几个\texttt{IntoIterator}的实现,分别是为共享引用(\texttt{\&T})、可变引用(\texttt{\&mut T})、移动(\texttt{T})提供的实现:
\begin{itemize}
    \item 给定一个集合的\emph{共享引用(shared reference)},\texttt{into\_iter}返回一个产生item的共享引用的迭代器。例如,在上面的代码中,\texttt{(\&favorites).into\_iter()}将会返回一个\texttt{Item}类型是\texttt{\&String}的迭代器。
    \item 给定一个集合的\emph{可变引用(mutable reference)},\texttt{into\_iter}返回一个产生item的可变引用的迭代器。例如,如果\texttt{vector}是\texttt{Vec<String>},那么\texttt{(\&mut vector).into\_iter()}将返回一个\texttt{Item}类型是\texttt{\&mut String}的迭代器。
    \item 当集合\emph{以值}传递时,\texttt{into\_iter}返回一个获取集合所有权并返回item自身的迭代器;item的所有权从集合移动到消费者,原来的集合在这个过程中被消耗。例如,上面代码中的\texttt{favorites.into\_iter()}会返回一个产生每个字符串值的迭代器;消费者会接受每个字符串的所有权。当迭代器被drop时,\texttt{BTreeSet}中剩余的所有元素也都会被drop,并且集合会变为未初始化。
\end{itemize}

因为一个\texttt{for}循环会对操作数调用\texttt{IntoIterator::into\_iter},这三种实现会导致有下面三种迭代方式:迭代集合的共享引用、迭代集合的可变引用、或者消耗集合并获取它的元素的所有权:
\begin{minted}{Rust}
    for element in &collection { ... }
    for element in &mut collection { ... }
    for element in collection { ... }
\end{minted}

这三种写法会调用上面列出的\texttt{IntoIterator}实现之一。

并不是每个类型都提供了全部这三种实现。例如,\texttt{HashSet}、\texttt{BTreeSet}、\texttt{BinaryHeap}没有实现共享引用的\texttt{IntoIterator},因为修改它们的元素可能会破坏类型的不变量:修改后的值可能会有不同的哈希值、或者和它的邻居的顺序关系会改变,因此修改元素会导致它们北方在错误的地方。其他的类型支持可变性,但只支持部分。例如,\texttt{HashMap}和\texttt{BTreeMap}产生表项的value的可变引用,以及key的共享引用,原因和上面类似。

一般的准则是迭代应该高效和可预测,因此Rust不提供开销很大或者可能展现出令人惊讶的行为的实现(例如,重新哈希被修改的\texttt{HashSet}条目并因此导致之后的迭代中可能再次遇到它们)。

切片实现了三种\texttt{IntoIterator}变体中的两个;因为它们并不拥有自己引用的元素,因此没有“以值”的实现。作为代替,\texttt{\&[T]}和\texttt{\&mut [T]}的\texttt{into\_iter}返回一个产生共享引用和可变引用的迭代器。如果你把底层切片类型\texttt{[T]}想象成一种集合,那么它就落入了之前的模式。

你可能已经注意到前两种\texttt{IntoIterator}的变体产生共享和可变的引用,这和调用\texttt{iter}或者\texttt{iter\_mut}是等价的。为什么Rust同时提供两者?

\texttt{IntoIterator}让\texttt{for}循环能正常工作,因此它显然是必要的。但当你不使用\texttt{for}循环时,使用\texttt{favorites.iter()}比\texttt{(\&favorites).into\_iter()}更加清晰。你可能会频繁需要以共享引用迭代,因此\texttt{iter}和\texttt{iter\_mut}也很有用。

\texttt{IntoIterator}在泛型代码中也很有用:你可以使用一个约束例如\texttt{T: IntoIterator}来限制类型参数\texttt{T}必须是可以迭代的类型。或者,你可以写\texttt{T: IntoIterator<Item=U>}来进一步要求迭代会产生\texttt{U}类型的值。例如,这个函数打印出任何item可以用\texttt{"{:?}"}格式打印的可迭代对象:
\begin{minted}{Rust}
    use std::fmt::Debug;

    fn dump<T, U>(t: T)
        where T: IntoIterator<Item=U>,
              U: Debug
    {
        for u in t {
            println!("{:?}", u);
        }
    }
\end{minted}
你不能在这个泛型函数中使用\texttt{iter}或者\texttt{iter\_mut},因为它们不是任何trait的方法:大多数可迭代类型只是恰好有这两个方法。

\subsection{\texttt{from\_fn}和\texttt{successor}}

\section{迭代器适配器}

\subsection{\texttt{fuse}}\label{fuse}

\section{消耗迭代器}

\section{实现自己的迭代器}