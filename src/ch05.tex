\chapter{引用}\label{ch05}

\emph{Libraries cannot provide new inabilities.}

\begin{flushright}
——Mark Miller
\end{flushright}

我们至今为止见过的所有指针类型——简单的\texttt{Box<T>}堆指针、\texttt{String}和\texttt{Vec}内部的指针都拥有值:当所有者被drop时,指针指向的值也会随之消失。Rust还有非拥有指针类型,称为\emph{引用},引用对指向的值的生命周期没有影响。

事实上,正相反,引用绝不应该比它们指向的值活的更长。你必须在你的代码中明确表明,引用的生命寿命比它指向的值更短。为了强调这一点,Rust将创建某个值的引用称为\emph{借用}值:你最终必须把你借走的还给它的所有者。

如果你在读到“你必须在你的代码中明确表明”时感到一丝怀疑,那说明你很优秀。引用自身并没有什么特殊的——本质上,它们只是地址。但保证它们安全的规则是Rust独有的,你以前不可能看到过类似的。尽管这些规则是Rust里最难掌握的部分,但它们能防止的经典的、日常的bug的范围之广令人惊讶,它们对多线程的影响也正在显现。这也是Rust的赌注。

这一章中,我们将讨论Rust中的引用如何工作,展示引用、函数和自定义类型如何包含生命周期信息来保证它们被安全使用,阐释它怎么能在编译期、不引入运行时开销的同时防止常见的bug。

\section{值的引用}

举个例子,假设我们要为文艺复兴时期优秀的艺术家和他们的著名作品建一个表格。Rust的标准库包含一个哈希表类型,所以我们可以像这样定义我们的类型:
\begin{minted}{Rust}
    use std::collections::HashMap;

    type Table = HashMap<String, Vec<String>>;
\end{minted}

换句话说,这是一个把\texttt{String}值映射到\texttt{Vec<String>}值的哈希表,它把艺术家的名字关联到它们的作品的名字。你可以使用\texttt{for}循环来迭代\texttt{HashMap}的条目,因此我们可以写一个函数打印出一个\texttt{Table}:
\begin{minted}{Rust}
    fn show(table: Table) {
        for (artist, works) in table {
            println!("works by {}:", artist);
            for work in works {
                println!("  {}", work);
            }
        }
    }
\end{minted}

构造和打印表格都很直观:
\begin{minted}{Rust}
    fn main() {
        let mut table = Table::new();
        table.insert("Gesualdo".to_string(),
                     vec!["many madrigals".to_string(),
                          "Tenebrae Responsoria".to_string()]);
        table.insert("Caravaggio".to_string(),
                     vec!["The Musicians".to_string(),
                          "The Calling of St. Matthew".to_string()]);
        table.insert("Cellini".to_string(),
                     vec!["Perseus with the head of Medusa".to_string(),
                          "a salt cellar".to_string()]);
        show(table);
    }
\end{minted}

它也能正常工作:
\begin{minted}{text}
    $ cargo run
         Running `/home/jimb/rust/book/fragments/target/debug/fragments`
    works by Gesualdo:
      many madrigals
      Tenebrae Responsoria
    works by Cellini:
      Perseus with the head of Medusa
      a salt cellar
    works by Caravaggio:
      The Musicians
      The Calling of St. Matthew
\end{minted}

但如果你阅读过上一章中有关move的小节,你就会发现\texttt{show}的定义有一些问题。首先,\texttt{HashMap}不是\texttt{Copy}类型——它不可能是,因为它持有动态分配的表格。因此当程序调用\texttt{show(table)}时,整个结构都被移动到函数里,变量\texttt{table}将变为未初始化。(迭代它时没有特定的顺序,你可能会得到一个不同的顺序,不用担心)如果调用者代码尝试继续使用\texttt{table},它会遇到问题:
\begin{minted}{Rust}
    ...
    show(table);
    assert_eq!(table["Gesualdo"][0], "many madrigals");
\end{minted}

Rust会报错\texttt{table}不再可用:
\begin{minted}{text}
    error: borrow of moved value: `table`
       |
    20 |     let mut table = Table::new();
       |         --------- move occurs because `table` has type
       |                   `HashMap<String, Vec<String>>`,
       |                   which does not implement the `Copy` trait
    ...
    31 |     show(table);
       |          ----- value moved here
    32 |     assert_eq!(table["Gesualdo"][0], "many madrigals");
       |                ^^^^^ value borrowed here after move
\end{minted}

事实上,如果我们仔细查看\texttt{show}的定义,会发现外层的\texttt{for}循环获取了哈希表的所有权然后完全消费了它,内层的\texttt{for}循环对每一个vector做了同样的事(我们之前已经在“liberté, égalité, fraternité”的例子中见过这种行为了)。因为move语义,我们仅仅是为了打印它就已经完全销毁了整个结构体。感谢你,Rust!

正确的处理方式是使用引用。引用让你可以访问一个值,同时不影响它的所有权。引用有两种:
\begin{itemize}
    \item \emph{共享引用}让你能读取但不能修改被引用的值。然而,你可以同时持有多个共享引用。表达式\texttt{\&e}返回一个指向\texttt{e}的值的共享引用,如果\texttt{e}的类型是\texttt{T},那么\texttt{\&e}的类型就是\texttt{\&T},读作“ref \texttt{T}”。共享引用是\texttt{Copy}类型。
    \item 如果你有一个值的\emph{可变引用},你可以读取和修改这个值。然而,你不能同时再有任何其他有效的引用。表达式\texttt{\&mut e}返回一个指向\texttt{e}的值的可变引用,它的类型是\texttt{\&mut T},读作“ref mute \texttt{T}”。可变引用不是\texttt{Copy}类型。
\end{itemize}

你可以将共享和可变引用的区别看作是一种在编译期强制\emph{多个读者或一个写者}的规则的方法。事实上,这个规则不仅适用于引用,还适用于被借用的值的所有者。只要一个值有共享引用存在,就算是它的拥有者也不能修改它,这个值被锁定了。当\texttt{show}正在使用\texttt{table}时没有人可以修改\texttt{table}。类似的,当一个值有可变引用时,它会排斥其他所有对这个值的方法,此时你不能使用值的持有者,直到可变引用消失。将共享和可变完全分离开来是内存安全的基础,我们将会在下一章介绍原因。

我们的示例中的打印函数不需要修改表格,只需要读取它的内容。所以调用者可以以共享引用的方式传递表格,像下面这样:
\begin{minted}{Rust}
    show(&table);
\end{minted}

引用是非拥有指针,所以\texttt{table}变量仍然保留着整个数据结构的所有权,\texttt{show}只是借用了它。当然,我们还需要调整\texttt{show}的定义来进行匹配,但你必须仔细看才能看出其中的区别:

\begin{minted}{Rust}
    fn show(table: &Table) {
        for (artist, works) in table {
            println!("works by {}:", artist);
            for work in works {
                println!("  {}", work);
            }
        }
    }
\end{minted}

\texttt{show}的参数\texttt{table}的类型从\texttt{Table}变为了\texttt{\&Table}:不再以值传递表格(会导致所有权移动到函数里),我们现在传递一个共享引用。这就是表面上唯一的变化。但是当我们执行函数体时到底是怎么工作的?

我们原先的版本\texttt{for}循环会获取\texttt{HashMap}的所有权按并消耗它。在我们的新版本中它接受一个\texttt{HashMap}的共享引用,迭代一个\texttt{HashMap}的共享引用被定义为产生每个条目的key和value的引用:\texttt{artist}从一个\texttt{String}变为\texttt{\&string},\texttt{works}从一个\texttt{Vec<String>}变为\texttt{\&Vec<String>}。

内部的循环也有类似的变化。迭代一个vector的共享引用被定义为产生它的每个元素的共享引用,因此\texttt{work}现在是一个\texttt{\&String}。这个函数里不再有任何所有权的变化,而是传递各种无所有权的引用。

现在,如果我们想写一个函数来把每个艺术家的作品按字母顺序排列,一个共享引用显然不够,因为共享引用不允许修改。排序函数需要表格的可变引用:
\begin{minted}{Rust}
    fn sort_works(table: &mut Table) {
        for (_artist, works) in table {
            works.sort();
        }
    }
\end{minted}

然后我们需要传递:
\begin{minted}{Rust}
    sort_works(&mut table);
\end{minted}

这个可变借用授予了\texttt{sort\_works}读取和修改我们的数据结构的能力,以满足vector的\texttt{sort}方法的需要。

当我们以会把所有权移动到函数中的方式传参时,我们称这种方式为\emph{以值}传参。如果我们传递值的引用,我们称之为\emph{以引用}传参。例如,我们通过将以值传参修改为以引用传参修复了\texttt{show}函数。许多语言都有这种区别,但它在Rust中尤其重要,因为它阐明了所有权如何被引用影响。

\section{使用引用}

上面的例子展示了引用的一个经典引用:允许函数在不获取所有权的情况下访问或者操作一个数据结构。但引用要更加灵活,让我们通过一些例子来进行更深入的了解。

\subsection{Rust的引用 vs C++的引用}
如果你熟悉C++的引用,你会发现它和Rust中的引用有很多相似之处。最重要的是,它们在机器层面都只是地址。但在实践中,Rust的引用使用起来有一种不同的感觉。

在C++中,引用通常通过隐式转换来创建,然后隐式解引用:
\begin{minted}{Rust}
    // C++代码!
    int x = 10;
    int &r = x;         // 初始化时隐式创建引用
    assert(r == 10);    // 隐式解引用r来访问x的值
    r = 20;             // 把20存储到x中,r仍然指向x
\end{minted}

在Rust中,引用通过\texttt{\&}运算符显示创建,通过\texttt{\*}运算符显式解引用:
\begin{minted}{Rust}
    // 从现在开始回到Rust的代码。
    let x = 10;
    let r = &x;         // &x是一个x的共享引用
    assert!(*r == 10);  // 显式解引用r
\end{minted}

要想创建可变引用,使用\texttt{\&mut}运算符:
\begin{minted}{Rust}
    let mut y = 32;
    let m = &mut y;     // &mut y是y的一个可变引用
    *m += 32;           // 显式解引用m来访问y的值
    assert!(*m == 64);  // 查看y的新值
\end{minted}

但是你可能回想起来,我们修改\texttt{show}函数让它以引用获取艺术家的表格时,我们从来没有使用过\texttt{*}运算符。这是为什么呢?

因为引用在Rust中使用如此广泛,所以如果需要的话,\texttt{.}运算符会隐式解引用它左侧的操作数:
\begin{minted}{Rust}
    struct Anime { name: &'static str, bechdel_pass: bool };
    let aria = Anime { name: "Aria: The Animation", bechdel_pass: true };
    let anime_ref = &aria;
    assert_eq!(anime_ref.name, "Aria: The Animation");

    // 等价于上面的代码,但显式写出了解引用
    assert_eq!((*anime_ref).name, "Aria: The Animation");
\end{minted}

\texttt{show}函数里使用的\texttt{println!}宏会展开为使用\texttt{.}运算符的代码,因此它也利用了这种隐式解引用。

如果方法调用需要的话,\texttt{.}运算符还会隐式借用左侧操作数的引用。例如,\texttt{Vec}的\texttt{sort}方法会获取vector的可变引用,因此下面的两个调用时等价的:
\begin{minted}{Rust}
    let mut v = vec![1973, 1968];
    v.sort();           // 隐式借用v的可变引用
    (&mut v).sort();    // 等价写法,不过更详细
\end{minted}

简而言之,C++在引用和左值(指向某个内存地址的表达式)之间隐式转换,如果需要这种转换会在任何地方出现;而在Rust中使用\texttt{\&}和\texttt{*}运算符来创建和解引用,除了\texttt{.}运算符是个例外,它会自动隐式借用和解引用。












