\chapter{外部函数}\label{ch23}
\emph{Cyberspace. Unthinkable complexity. Lines of lightranged in the non-space of the mind, clusters andconstellations of data. Like city lights, receding . . .}

\begin{flushright}
    ——William Gibson, \emph{Neuromancer}
\end{flushright}

不幸的是,不是世界上每个程序都是用Rust编写的。我们可能想在我们的Rust程序中使用很多用其他语言实现的优秀的库和接口。Rust的\emph{外部语言接口(foreign function interface(FFI))}让Rust代码能调用C编写的函数,和一部分C++编写的函数。因为大多数操作系统提供C接口,Rust的外部函数接口允许直接访问任何类型的底层设施。

在本章中,我们将编写一个链接到\texttt{libgit2}的程序,它是一个用于Git版本控制系统的C库。首先,我们将展示怎么直接在Rust中使用C函数,就使用我们上一章中介绍的unsafe特性。然后,我们将展示如何构建\texttt{liggit2}的safe接口,借助开源的\texttt{git2-rs} crate的灵感,这个crate正好实现了这一点。

我们将假设你熟悉C语言和编译链接C程序的机制。处理C++也相差不多。我们还假设你熟悉Git版本控制系统。

确实有Rust crate用于和很多其他语言例如Python、JavaScript、Lua和Java交互。我们没有足够的篇幅来介绍它们,但所有这些接口最终都是使用C外部函数接口实现的,因此无论你想和什么语言交互,这一章都能给你一个开头。

\section{寻找公共的数据表示}\label{repr}
Rust和C的公共基础是机器语言,因此为了预测Rust的值在C代码中看起来是什么样的,或者反过来,你需要考虑它们的机器级表示。在整本书中,我们一直着重展示一个值在内存中的实际表示,因此你可能已经注意到了C和Rust的数据世界有很多共通之处:例如一个Rust的\texttt{usize}和C的\texttt{size\_t}是相同的,两门语言中的结构体也基本相同。为了建立起Rust和C中相应类型的关系,我们将从基本类型开始,并逐渐扩展到更复杂的类型。

鉴于C主要用作系统编程,C中类型的表示总是令人惊讶的宽松:例如一个\texttt{int}通常是32位,但可能会更长,或者短到16位;一个C的\texttt{char}可能是有符号的也可能是无符号的。为了应对这种可变性,Rust的\texttt{std::os::raw}模块定义了一些Rust的类型,这些类型保证和相应的C类型有完全相同的表示(\autoref{t23-1})。其中包括基本的整数和字符类型。

\begin{table}[htbp]
    \centering
    \caption{\texttt{std::os::raw}中的Rust类型}
    \label{t23-1}
    \begin{tabular}{ll}
        \textbf{C类型}  &   \textbf{相应的\texttt{std::os::raw}类型}    \\
        \hline
        \texttt{short}          &   \texttt{c\_short}   \\
        \rowcolor{tablecolor}
        \texttt{int}            &   \texttt{c\_int}     \\
        \texttt{long}           &   \texttt{c\_long}    \\
        \rowcolor{tablecolor}
        \texttt{long long}      &   \texttt{c\_longlong}    \\
        \texttt{unsigned short} &   \texttt{c\_ushort}  \\
        \rowcolor{tablecolor}
        \texttt{unsigned, unsigned int} &   \texttt{c\_uint}    \\
        \texttt{unsigned long}  &   \texttt{c\_ulong}   \\
        \rowcolor{tablecolor}
        \texttt{unsigned long long}     &   \texttt{c\_ulonglong}   \\
        \texttt{char}           &   \texttt{c\_char}    \\
        \rowcolor{tablecolor}
        \texttt{signed char}    &   \texttt{c\_schar}   \\
        \texttt{unsigned char}  &   \texttt{c\_uchar}   \\
        \rowcolor{tablecolor}
        \texttt{float}           &   \texttt{c\_float}    \\
        \texttt{double}         &   \texttt{c\_double}  \\
        \rowcolor{tablecolor}
        \texttt{void *, const void *}   &   \texttt{*mut c\_void, *const c\_void}   \\
    \end{tabular}
\end{table}

有关\autoref{t23-1}的注意事项:
\begin{enumerate}
    \item 除了\texttt{c\_void}之外,这里所有的Rust类型都是某些基本Rust类型的别名:例如\texttt{c\_char}是\texttt{i8}或者\texttt{u8}。
    \item Rust的\texttt{bool}等价于C或C++的\texttt{bool}。
    \item Rust的32位\texttt{char}类型并不等同于\texttt{wchar\_t},后者的宽度和编码取决于具体实现。C的\texttt{char32\_t}倒是更接近一点,但它的编码仍然并不保证是Unicode。
    \item Rust的基础\texttt{usize}和\texttt{isize}类型和C的\texttt{size\_t}和\texttt{ptrdiff\_t}有相同的表示。
    \item C/C++指针和C++的引用对应Rust的原始指针类型\texttt{*mut T}和\texttt{*const T}。
    \item 从技术上讲,C标准允许实现使用一些Rust中没有相应类型的表示:36位整数、用符号+数字来表示有符号值等等。在实践中,在Rust被移植到的每个平台上,每个基本的C整数类型在Rust中都有对应的类型。
\end{enumerate}

为了定义兼容C结构体的Rust结构体类型,你可以使用\texttt{\#[repr(C)]}属性。在结构体的定义上面放上\texttt{\#[repr(C)]}可以要求Rust按照C编译器的方式放置结构体中的字段。例如,\texttt{libgit2}的\emph{git2/errors.h}头文件定义了下面的C结构体来提供一个错误的详情:
\begin{minted}{C}
    typedef struct {
        char *message;
        int klass;
    } git_error;
\end{minted}

你可以按照下面这样定义一个内存表示完全相同的Rust类型:
\begin{minted}{Rust}
    use std::os::raw::{c_char, c_int};

    #[repr(C)]
    pub struct git_error {
        pub message: *const c_char,
        pub klass: c_int
    }
\end{minted}

\texttt{\#[repr(C)]}属性只影响struct自身的布局,不会影响单个字段的表示,因此为了和C struct匹配,每一个字段也都要使用C风格的类型:例如用\texttt{*const c\_char}替换\texttt{char *},用\texttt{c\_int}替换\texttt{int}。

在这个特定的例子中,\texttt{\#[repr(C)]}属性可能并不会改变\texttt{git\_error}的布局。因为实际上没有那么多放置一个指针和一个整数的方法。但C和C++都保证一个结构体的成员按照声明的顺序依次在内存中排布,而Rust会按照总结构体大小最小的方式来组织字段,并且0大小的类型不占用空间。\texttt{\#[repr(C)]}属性告诉Rust按照C的规则来布局。

你也可以使用\texttt{\#[repr(C)]}来控制C风格的enum的表示:
\begin{minted}{Rust}
    #[repr(C)]
    #[allow(non_camel_case_types)]
    enum git_error_code {
        GIT_OK          = 0,
        GIT_ERROR       = -1,
        GIT_ENOTFOUND   = -3,
        GIT_EEXISTS     = -4,
        ...
    }
\end{minted}

通常情况下,Rust在选择如何表示enum时会使用各种技巧。例如,我们提到过Rust在一个单字中存储\texttt{Option<\&T>}(如果\texttt{T}是sized)。如果没有\texttt{\#[repr(C)]},Rust会使用单个字节来表示\texttt{git\_error\_code} enum;有了\texttt{\#[repr(C)]}之后,Rust会和C一样用一个C \texttt{int}一样大的值来存储。

你可以要求Rust使用和某些整数相同的表示来存储enum。上面的定义如果以\texttt{\#[repr(i16)]}开头,将会得到一个和下面C++ enum相同的16位的表示:
\begin{minted}{C++}
    #include <stdint.h>

    enum git_error_code: int16_t {
        GIT_OK          = 0,
        GIT_ERROR       = -1,
        GIT_ENOTFOUND   = -3,
        GIT_EEXISTS     = -4,
        ...
    };
\end{minted}

正如之前提到的,\texttt{\#[repr(C)]}还可以用于union。\texttt{\#[repr(C)]} union的字段总是从union的内存的第一个位(0偏移处)开始。

假设你有一个C struct使用一个union来存储一些数据,再用一个tag值来指示应该使用union的哪个值,类似于Rust的enum一样:
\begin{minted}{C}
    enum tag {
        FLOAT = 0,
        INT   = 1,
    };

    union number {
        float f;
        short i;
    };

    struct tagged_number {
        tag t;
        number n;
    };
\end{minted}

Rust可以通过对enum、struct和union类型都应用\texttt{\#[repr(C)]}来实现一个这样的结构体,然后使用\texttt{match}语句基于tag来选择一个struct中的union的字段:
\begin{minted}{Rust}
    #[repr(C)]
    enum Tag {
        Float = 0,m
        Int   = 1
    }

    #[repr(C)]
    union FloatOrInt {
        f: f32,
        i: i32,
    }

    #[repr(C)]
    struct Value {
        tag: Tag,
        union: FloatOrInt
    }

    fn is_zero(v: Value) -> bool {
        use self::Tag::*;
        unsafe {
            match v {
                Value { tag: Int, union: FloatOrInt { i: 0 } } => true,
                Value { tag: Float, union: FloatOrInt { f: num } } => (num == 0.0),
                _ => false
            }
        }
    }
\end{minted}

使用这种技术,即使是复杂的结构体也可以很容易地跨FFI边界使用。

在Rust和C之间传递字符串要稍微更难一点。C使用一个空结尾的字符数组的指针来表示字符串。另一边Rust显式地存储字符串的长度,要么是\texttt{String}的一个字段,要么是一个胖引用\texttt{\&str}的第二个字。Rust的字符串不是空字符结尾的;事实上,它们的内容里可能包含空字符,这些空字符和其他字符一样,没有区别。

这意味着你不能借用一个Rust字符串来当做C字符串:如果你传给C代码一个Rust字符串的指针,它可能错误地把内容中一个空字符当做字符串的结尾,或者越界去查找一个不存在的结尾空字符。从另一个方向考虑,你也许可以借用一个C字符串作为一个Rust的\texttt{\&str},只要它的内容是有效的UTF-8。

这种情况强迫Rust必须把C的字符串看做和\texttt{String}、\texttt{\&str}完全不同的类型。在\texttt{std::ffi}模块中,\texttt{CString}和\\
\texttt{CStr}类型表示拥有所有权的和借用的空字符结尾的字节数组。和\texttt{String}、\texttt{str}比起来,\texttt{CString}和\texttt{CStr}的方法非常有限,基本只有构建自身和转换成其他类型的方法。我们将在下一节中用实例展示这些类型。

\section{声明外部函数和变量}

\section{一个libgit2的安全接口}\label{SafeInter}