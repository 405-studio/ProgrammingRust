\chapter{错误处理}\label{ch07}

\emph{I knew if I stayed around long enough, something like this would happen.}

\begin{flushright}
    ——George Bernard Shaw on dying
\end{flushright}

Rust的错误处理不同寻常,无法用很短的一个章节来介绍它。其实它里面并没有什么困难的概念,只有一些可能对你来说可能很新的概念。这一章将覆盖Rust中两种不同的错误处理:panic和\texttt{Result}。

一般的错误使用\texttt{Result}类型来处理,\texttt{Result}通常代表程序之外的东西引起的问题,例如错误的输入、网络中断、权限问题等。这种情况的出现不由我们决定,即使是一个完全没有bug的程序也可能随时遇到它们。这一章中的大部分内容都是在讨论这种错误。我们将首先介绍panic,因为它比较简单。

panic是另一种错误,一种\emph{永远不应该发生}的错误。

\section{panic}

当程序遇到一些由程序自身的bug导致的非常糟糕的事情时它会panic。例如:
\begin{itemize}
    \item 数组访问越界
    \item 整数除以0
    \item 在值为\texttt{Err}的\texttt{Result}上调用\texttt{.expect()}方法
    \item 断言失败
\end{itemize}

(还有一个宏\texttt{panic!()},用于当你的代码自己发现了错误,想要直接触发panic的情况。\texttt{panic!()}接受可选的\texttt{println!()}-风格的参数,用于构建错误信息。)

这些条件的共同之处在于——它们都是程序员的错。一条好的经验法则是:“不要panic”。

但是我们都有犯错误的时候。当这些不该发生的错误发生了的时候,该怎么办?值得注意的是,Rust给了你一个选择:Rust可以展开堆栈或者中止进程。栈展开是默认行为。

\subsection{栈展开}
当海盗们瓜分抢来的战利品时,船长将得到一半的战利品。普通的船员们均分剩下的一半。(海盗们讨厌分数,因此如果均分时不能除尽,结果会向下取整,余数将分给船上的鹦鹉。)
\begin{minted}{Rust}
    fn pirate_share(total: u64, crew_size: usize) -> u64 {
        let half = total / 2;
        half / crew_size as u64
    }
\end{minted}

这段代码也许可以工作几个世纪,直到有一天船长是抢劫之后唯一的幸存者。如果我们传递的\texttt{crew\_size}为0,它将会除以0。在C++中,这将是未定义行为。在Rust中,它会触发panic,panic通常会按照如下方式继续:
\begin{itemize}
    \item 打印一条错误消息到终端:
    \begin{minted}{text}
    thread 'main' panicked at 'attempt to divide by zero',
    pirates.rs:3780
    note: Run with `RUST_BACKTRACE=1` for a backtrace.
    \end{minted}

    如果你设置了\texttt{RUST\_BACKTRACE}环境变量,Rust还会打印出此时的堆栈信息。

    \item 堆栈被展开。这和C++中的异常处理很像。
    
    任何当前函数内的临时值、局部变量、或者参数都会按照与它们创建时相反的顺序被drop掉。

    drop一个值意味着清理它:函数使用过的任何\texttt{String}或\texttt{Vec}都会被释放,任何打开的\texttt{File}都会被关闭,等等。用户自定义的\texttt{drop}方法也会被调用,见“\hyperref[drop]{Drop}”一节。在\texttt{pirate\_share()}的例子中,没有要清理的内容。

    一旦当前的函数调用被清理完毕,我们会移动到它的调用者,以同样的方式drop它的变量和参数。然后我们移动到\emph{那个}函数的调用者,以此类推。

    \item 最后,线程退出。如果panic的线程是主线程,整个进程会退出(退出代码不为0)。
\end{itemize}

对这种有序的处理,也许\emph{panic}是一个有误导性的名字。panic并不是崩溃,也不是未定义行为,它更类似于Java中的\texttt{RuntimeException}或C++中的\texttt{std::logic\_error}。它的行为都是良定义的,它只是不应该发生。

panic是安全的。它不违背Rust中的任何安全规则,即使你设法在一个标准库的方法中引起panic,它也用于不会导致悬垂指针或者初始化到一半的值。关键在于Rust在任何错误的事情发生之前就捕捉到了无效的数组访问或者类似的情况。如果继续下去将是不安全的,所以Rust会展开堆栈。但进程的其他部分可以继续运行。

panic是以线程为单位。一个线程可以panic,而其他线程继续处理它们的业务。在\hyperref[ch19]{第19章}中,我们会展示一个父线程怎么查明一个子线程是否panic并优雅地处理错误。

还有一种方式\emph{捕获}栈展开,允许线程存活并继续运行。标准库函数\texttt{std::panic::catch\_unwind()}可以做到这一点。我们不会解释如何使用它,但Rust的测试工具使用了这个机制,用于在测试时断言失败的情况下恢复执行(当编写可以在C或C++中调用的Rust代码时这也是必须的,因为在非Rust代码中的栈展开是未定义行为,见\hyperref[ch22]{第22章})。

理想情况下,我们希望没有bug并且永远不会panic的代码。但没有完美的事物,你可以使用线程和\texttt{catch\_unwind()}来处理panic,让你的程序更加健壮。一个重要的警告是这些工具只会捕获展开堆栈的panic。不是所有的panic都以这种方式处理。

\subsection{中止}
栈展开是默认的panic行为,但还有两种情况下Rust不会尝试展开堆栈。

如果在Rust尝试清理时一个\texttt{.drop()}方法触发了第二次panic,Rust会认为这是致命错误,它会停止栈展开并中止整个进程。

还有,Rust的panic行为可以自定义。如果你以参数\texttt{-C panic=abort}编译,程序中的\emph{第一个}panic会立即中止进程。(这个选项下,Rust不需要知道如何展开堆栈,因此可以减小编译出的代码的体积。)

最后总结一下关于Rust中panic的讨论。没有更多要说的了,因为普通的Rust代码没有义务处理panic。即使你使用了线程或\texttt{catch\_unwind()},所有处理panic的代码很可能会集中在少数部分。没有理由检查程序中的每一个函数然后预测并处理里面的bug。其他因素导致的错误则是另一码事。

\section{Result}

Rust里没有异常,可能会失败的函数可以通过返回\texttt{Result}来表达类似的含义:
\begin{minted}{Rust}
    fn get_weather(location: LatLng) -> Result<WeatherReport, io::Error>
\end{minted}
\texttt{Result}类型表明可能会失败。当调用\texttt{get\_weather()}函数时,它可能会返回\emph{成功的结果}\texttt{Ok(weather)},其中\texttt{weather}是一个新的\texttt{WeatherReport}值;或者返回一个\emph{错误的结果}\texttt{Err(error\_value)},其中\texttt{error\_value}是一个解释错误的\texttt{io::Error}。

Rust要求我们每次调用这个函数时都要进行一些错误处理。我们必须对返回的\texttt{Result}做\emph{一些处理},才能得到\texttt{WeatherReport}值。如果\texttt{Result}值没有被使用的话编译器也会警告。

在\hyperref[ch10]{第10章}中,我们将看到标准库是怎么定义\texttt{Result}的、以及你该怎么自己定义类似的类型。现在,我们将专注于如何使用\texttt{Result}来进行错误处理。我们将看到如何捕捉、传播和报告错误,以及一些组织和使用\texttt{Result}类型的常见模式。

\subsection{捕捉错误}
最通用的处理\texttt{Result}的方法就是我们在\hyperref[ch02]{第2章}中展示的:使用\texttt{match}表达式。
\begin{minted}{Rust}
    match get_weather(hometown) {
        Ok(report) => {
            display_weather(hometown, &report);
        }
        Err(err) => {
            println!("error querying the weather: {}", err);
            schedule_weather_retry();
        }
    }
\end{minted}

这在Rust中等价于其它语言的\texttt{try/catch}。当你想要自己处理错误,不把错误传递给调用者时你可以使用这种方式。

\texttt{match}有些繁琐,所以\texttt{Result<T, E>}提供了一些在常见场景下很有用的方法。这些方法中的每一个的内部实现都用到了\texttt{match}表达式。(\texttt{Result}的全部方法可以查询在线文档。这里列出的方法是我们最常使用的。)

\codeentry{result.is\_ok(), result.is\_err()}
\hangparagraph{返回一个\texttt{bool}值表示\texttt{result}是成功还是错误。}

\codeentry{result.ok()}
\hangparagraph{以\texttt{Option<T>}类型返回成功的结果。如果\texttt{result}是一个成功的结果,会返回\texttt{Some(success\_value)};否则会返回\texttt{None},丢弃错误的值。}

\codeentry{result.err()}
\hangparagraph{以\texttt{Option<E>}类型返回错误的值。}

\codeentry{result.unwrap\_or(fallback)}
\hangparagraph{如果\texttt{result}是成功的结果的话,返回成功的值。否则,返回\texttt{fallback},丢弃错误的值。}

\begin{minted}{Rust}
    // 南加州通常的天气情况。
    const THE_USUAL: WeatherReport = WeatherReport::Sunny(72);

    // 获取真实的天气预报。
    // 如果失败,倒退到通常的情况。
    let report = get_weather(los_angeles).unwrap_or(THE_USUAL);

    display_weather(los_angeles, &report);
\end{minted}

\hangparagraph{这是\texttt{.ok()}的一个漂亮的替代,因为返回的类型是\texttt{T}而不是\texttt{Option<T>}。当然,只有当存在有意义的fallback值时这么写才有用。}

\codeentry{result.unwrap\_or\_else(fallback\_fn)}
\hangparagraph{这和上一个类似,但不再是直接传递fallback值,而是传递一个函数或者闭包。这通常用于计算fallback时会很浪费时间的情况,因为只有当结果是错误时,\texttt{fallback\_fn}才会被调用。}

\begin{minted}{Rust}
    let report =
        get_weather(hometown)
        .unwrap_or_else(|_err| vague_prediction(hometown));
\end{minted}

\hangparagraph{\hyperref[ch14]{第14章}会详细介绍闭包。}

\codeentry{result.unwrap()}
\hangparagraph{如果\texttt{result}是成功的结果,会返回成功的值。然而,如果\texttt{result}是错误的结果,这个方法会panic。这个方法有它的用途,我们将在之后讨论。}

\codeentry{result.expect(message)}
\hangparagraph{和\texttt{.unwrap()}基本相同,不过你可以提供一条panic时打印的错误信息。}

最后,还有一些和引用相关的方法:
\codeentry{result.as\_ref()}
\hangparagraph{把一个\texttt{Result<T, E>}转换为\texttt{Result<\&T, \&E>}。}

\codeentry{result.as\_mut()}
\hangparagraph{和上面类似,但借用可变引用。返回类型是\texttt{Result<\&mut T, \&mut E>}。}

最后两个方法很有用的一个原因是这里列出的其他所有方法,除了\texttt{.is\_ok()}和\texttt{.is\_err()}之外,都会\emph{消耗}操作的\texttt{result}值。也就是说,它们以值获取\texttt{self}参数。有时如果能访问\texttt{result}里的值而不破坏它也会带来便利,这正是\texttt{.as\_ref()}和\texttt{.as\_mut()}做的事情。例如,假设你想调用\texttt{result.ok()},但你需要\texttt{result}保持完好。你可以写\texttt{result.as\_ref().ok()},它只会借用\texttt{result},返回\texttt{Option<\&T>}而不是\texttt{Option<T>}。

\subsection{Result类型别名}
有时你会看到Rust的文档中看起来像是省略了\texttt{Result}的错误类型:
\begin{minted}{Rust}
    fn remove_file(path: &Path) -> Result<()>
\end{minted}
这意味着这里使用了一个\texttt{Result}类型的别名。

类型别名是一种缩写的类型名。模块通常会定义一个\texttt{Result}类型的类型别名来避免在模块内几乎每个函数中都重复书写相同的错误类型。例如,标准库的\texttt{std::io}模块就包含这样一行代码:
\begin{minted}{Rust}
    pub type Result<T> = result::Result<T, Error>;
\end{minted}

这里定义了一个公有类型\texttt{std::io::Result<T>}。它是\texttt{Result<T, E>}的一个别名,把\texttt{std::io::Error}硬编码为错误类型。在实践中,这意味着如果你写了\texttt{use std::io},Rust将会把\texttt{io::Result<String>}看作\texttt{Result<String, io::Error>}的缩写。

当类似于\texttt{Result<()>}的东西出现在在线文档里的时候,你可以点击标识符\texttt{Result}来查看使用了哪个类型别名和具体的错误类型。在实践中,通常能从上下文中明显地推断出错误类型。

\subsection{打印错误}
有时处理错误的唯一方法就是把它打印到终端然后继续运行程序。我们已经展示了一种这样做的方法:
\begin{minted}{Rust}
    println!("error querying the weather: {}", err);
\end{minted}

标准库定义了几个错误类型:\texttt{std::io::Error}、\texttt{std::fmt::Error}、\texttt{std::str::Utf8Error}等等。它们全都实现了一个公共接口\texttt{std::error::Error} trait,这意味着它们共享下面的特性和方法:

\codeentry{println!()}
\hangparagraph{所有的错误类型都可以用它来打印。使用\texttt{\{\}}格式说明符只会显式一条简短的错误消息。作为替代,你可以使用\texttt{\{:?\}}格式说明符来获取错误的\texttt{Debug}视图。这对用户不是很友好,但包含更多的额外信息。}
\begin{minted}{text}
    // result of `println!("error: {}", err);`
    error: failed to lookup address information: No address associated with hostname

    // result of `println!("error: {:?}", err);`
    error: Error { repr: Custom(Custom { kind: Other, error: StringError("failed to lookup address information: No address associated with hostname") }) }
\end{minted}

\codeentry{err.to\_string()}
\hangparagraph{返回一个\texttt{String}类型的错误消息。}

\codeentry{err.source()}
\hangparagraph{返回一个导致\texttt{err}的底层错误,如果有的话。例如,一个网络错误可能会导致一次银行交易失败,这可能反过来导致你的船被收回。如果\texttt{err.to\_string()}是\texttt{"boat was repossessed"},那么\texttt{err.source()}可能返回一个有关失败的交易的错误。那个错误的\texttt{.to\_string()}可能是\texttt{"failed to transfer \$300 to United Yacht Supply"},它的\texttt{.source()}可能是一个有关网络中断导致错误的详情的\texttt{io::Error}。第三个错误就是根源,因此它的\texttt{.source()}方法会返回\texttt{None}。因为标准库只包含低级的特性,所以标准库中的错误的来源通常都是\texttt{None}。}

打印一个错误值并不会打印出它的来源。如果你想要打印出所有可用的信息,使用这个函数:
\begin{minted}{Rust}
    use std::error::Error;
    use std::io::{Write, stderr};

    /// Dump an error message to `stderr`
    ///
    /// If another error happens while building the error message or
    /// writing to `stderr`, it is ignored.
    fn print_error(mut err: &dyn Error) {
        let _ = writeln!(stderr(), "error: {}", err);
        while let Some(source) = err.source() {
            let _ = writeln!(stderr(), "caused by: {}", source);
            err = source;
        }
    }
\end{minted}

\texttt{writeln!}宏类似于\texttt{println!}宏,除了它把数据写入你指定的流。这里,我们将错误消息写入到了标准错误流\texttt{std::io::stderr}。我们也可以使用\texttt{eprintln!}宏来做同样的事,不过如果中途出错\texttt{eprintln!}会panic。在\texttt{print\_error}中,我们希望忽略写入消息时出现的错误;我们会在这一章中的“\hyperref[ignoreerr]{忽略错误}”一节中解释原因。

标准库的错误类型不包含堆栈追踪,不过当使用unstable版本的Rust编译器时,流行的\texttt{anyhow} crate提供了一个现成的有堆栈跟踪的错误类型(Rust 1.50版本中,标准库中捕获堆栈追踪的函数还没有被稳定化)。

\subsection{传播错误}\label{properror}
很多时候当我们尝试一些可能会失败的操作时,我们并不想立即捕捉并处理错误,在每一个可能出错的地方都使用10行的\texttt{match}表达式太过繁琐。

取而代之的是,当错误出现时我们通常希望让调用者处理它。我们把错误沿着调用栈往上\emph{传播}。

Rust有一个\texttt{?}运算符来做这个工作。你可以在任何产生\texttt{Result}的表达式后加上\texttt{?},例如返回\texttt{Result}的函数调用:
\begin{minted}{Rust}
    let weather = get_weather(hometown)?;
\end{minted}

\texttt{?}运算符的行为取决于函数返回成功的结果还是错误的结果:
\begin{itemize}
    \item 成功时,它会解包\texttt{Result}来获取成功的值。这里\texttt{weather}的类型并不是\texttt{Result<WeatherReport, io::Error>},而是\texttt{WeatherReport}。
    \item 错误时,它会立刻从函数里返回,把错误沿着调用链向上传递。为了做到这一点,\texttt{?}只能用于返回值是\texttt{Result}类型的函数里的\texttt{Result}值。
\end{itemize}

\texttt{?}运算符并没有魔法。你可以使用\texttt{match}表达式做到相同的事,只不过要繁琐很多:
\begin{minted}{Rust}
    let weather = match get_weather(hometown) {
        Ok(success_value) => success_value,
        Err(err) => return Err(err)
    };
\end{minted}

这和\texttt{?}运算符的唯一区别是一些涉及类型和转换的细节,我们将在下一节中讨论这些细节。

在旧代码中,你可能还会看到\texttt{try!()}宏,在Rust 1.13引入\texttt{?}运算符之前它是传播错误的通常方法:
\begin{minted}{Rust}
    let weather = try!(get_weather(hometown));
\end{minted}
这个宏会展开成类似于上面的\texttt{match}表达式。

我们很容易忘记程序中的错误究竟有多么无孔不入,尤其是那些与操作系统交互的代码。\texttt{?}运算符有时会在几乎每一行代码中出现:
\begin{minted}{Rust}
    use std::fs;
    use std::io;
    use std::path::Path;

    fn move_all(src: &Path, dst: &Path) -> io::Result<()> {
        for entry_result in src.read_dir()? {   // 打开目录可能会失败
            let entry = entry_result?;          // 读取目录可能会失败
            let dst_file = dst.join(entry.file_name());
            fs::rename(entry.path(), dst_file)?;    // 重命名可能会失败
        }
    }
\end{minted}

\texttt{?}也能用于\texttt{Option}类型。在一个返回\texttt{Option}的函数中,你可以使用\texttt{?}来解包值,并在为\texttt{None}的情况下直接返回:
\begin{minted}{Rust}
    let weather = get_weather(hometown).ok()?;
\end{minted}

\subsection{处理多种错误类型}
通常,可能会出错的不止一件事情。假设我们要从文本文件中读取一个数字:
\begin{minted}{Rust}
    use std::io::{self, BufRead};

    /// 从文本文件中读取一个证书。
    /// 文件中应该每行一个数字。
    fn read_numbers(file: &mut dyn BufRead) -> Result<Vec<i64>, io::Error> {
        let mut numbers = vec![];
        for line_result in file.lines() {
            let line = line_result?;     // 读取一行可能失败
            numbers.push(line.parse()?); // 解析整数可能失败
        }
        Ok(numbers)
    }
\end{minted}

Rust会报一个编译时错误:
\begin{minted}{text}
    error: `?` couldn't convert the error to `std::io::Error`

      numbers.push(line.parse()?);  // 解析整数可能失败
                               ^
                the trait `std::convert::From<std::num::ParseIntError>`
                is not implemented for `std::io::Error`

    note: the question mark operation (`?`) implicitly performs a conversion
    on the error value using the `From` trait
\end{minted}

当我们接触到\hyperref[ch11]{第11章}中有关trait的内容时才能明白错误消息中的术语。到现在为止,只要知道Rust是在说\texttt{?}运算符不能把\texttt{std::num::ParseIntError}转换为\texttt{std::io::Error}类型就够了。

这里的问题在于从文件中读取一行和解析整数会产生两种不同的错误类型。\texttt{line\_result}的类型是\texttt{Result<String, std::io::Error>}。\texttt{line.parse()}的类型是\texttt{Result<i64, std::num::ParseIntError>}。我们的\texttt{read\_numbers()}函数的返回类型只能容乃\texttt{io::Error}。Rust尝试把\texttt{ParseIntError}转换为\texttt{io::Error},但这样的转换并不存在,所以向我们汇报了一个错误。

有几种处理这个错误的方法。例如,我们在\hyperref[ch02]{第2章}中用于创建曼德勃罗集的图片文件的\texttt{image} crate中就定义了它自己的错误类型\texttt{ImageError},并实现了从\texttt{io::Error}和几种其他错误类型到\texttt{ImageError}的转换。如果你想要走这条路,可以尝试\texttt{thiserror} crate,它被设计的目的就是帮你只用少量代码定义良好的错误类型。

一个更简单的方法是使用Rust内建的机制。所有标准库里的错误类型都可以转换为类型\texttt{Box<dyn std::error::Error + Send + Sync + 'static>}。这有一些复杂,不过\texttt{dyn std::error::Error}代表“任何错误”,而\texttt{Send + Sync + 'static}使它可以安全地在线程间传递,这通常也是你想要的\footnote{你也可以考虑使用流行的\texttt{anyhow} crate,它提供类似于我们的\texttt{GenericError}和\texttt{GenericResult}的类型,不过还带有一些其他的有用的特性。}。为了方便,你可以定义类型别名:
\begin{minted}{Rust}
    type GenericError = Box<dyn std::error::Error + Send + Sync + 'static>;
    type GenericResult<T> = Result<T, GenericError>;
\end{minted}

然后,把\texttt{read\_numbers()}的返回类型修改为\texttt{GenericResult<Vec<i64>>}。修改之后,函数就可以编译了。\texttt{?}运算符会在需要的时候把任何错误类型转换为\texttt{GenericError}。

顺带,\texttt{?}运算符通过一个标准库方法来完成自动转换,你也可以手动使用这个方法。可以调用\texttt{GenericError::from()}来将任何错误转换为\texttt{GenericError}类型:
\begin{minted}{Rust}
    let io_error = io::Error::new(      // 创建一个我们自己的io::Error
        io::ErrorKind::Other, "timed out");
    return Err(GenericError::from(io_error)); // 手动转换为GenericError
\end{minted}

我们将在\hyperref[ch13]{第13章}中详细介绍\texttt{From} trait和它的\texttt{from()}方法。

用\texttt{GenericError}方法处理的缺点是返回类型将不能精确地告诉调用者到底发生了什么类型的错误。如果你要调用一个返回\texttt{GenericResult}的函数并且向处理某一种特性类型的错误,然而继续传播其它所有类型的错误,可以使用泛型方法\texttt{error.downcast\_ref::<ErrorType>()}。\emph{如果}这个错误正好是你指定的错误类型:
\begin{minted}{Rust}
    loop {
        match compile_project() {
            Ok(()) => return Ok(()),
            Err(err) => {
                if let Some(mse) = err.downcast_ref::<MissingSemicolonError>() {
                    insert_semicolon_in_source_code(mse.file(), mse.line())?;
                    continue;   // try again!
                }
                return Err(err);
            }
        }
    }
\end{minted}

许多语言都有类似这样的内建语法,但它们通常极少被用到。Rust使用一个方法来完成这件事。

\subsection{处理“不可能发生”的错误}

\subsection{忽略错误}\label{ignoreerr}

\subsection{在main()中处理错误}

\subsection{声明自定义错误类型}

\subsection{为什么选择Result}
