\chapter{枚举与模式}\label{ch10}

\emph{Surprising how much computer stuff makes sense viewed as tragic deprivation of sum types (cf. deprivation of lambdas).}

\begin{flushright}
    ——Graydon Hoare
\end{flushright}

这一章的第一个话题将是一个古老的、强有力的、可以帮你在短期内完成很多工作的、并且在许多语言中以不同的名字广为人知的特性。但它并不是魔鬼。而是一种用户自定义的数据类型,它是ML和Haskell程序员们熟知的和类型、也是互斥的联合、还是代数数据类型。在Rust中,它们被称为\emph{枚举(enumerations)},或者简写为\emph{enum}。和魔鬼不同的是,它们非常安全、索取的代价也很小。

C++和C\#都有枚举,你可以使用它们来定义自己的类型,这种类型的取值范围是一些命名常量的集合。例如,你可能定义过一个叫\texttt{Color}的类型,取值范围为\texttt{Red}、\texttt{Orange}、\texttt{Yellow}等等。这种枚举在Rust中也能工作,但Rust进一步扩展了枚举。一个Rust枚举可以包含数据,包括多种不同类型的数据。例如,Rust的\texttt{Result<String, io::Error}类型是一个枚举;这样一个值要么是一个包含\texttt{String}的\texttt{Ok}值要么是一个包含\texttt{io::Error}的\texttt{Err}值。这就超出了C++和C\#中枚举的能力。它更像C中的\texttt{union}——但和联合不同的是,Rust的枚举是类型安全的。

枚举适用于一个值有多种可能的情况。使用它们的“代价”是你必须使用模式匹配来安全地访问数据,这也是我们这一章中的第二个话题。

如果你使用过Python的解包或者JavaScript中的解构,那你可能觉得模式也很熟悉,但Rust同样扩展了模式。Rust的模式有点像匹配数据的正则表达式。它们被用来测试一个值是否具有特定的期望的形态。它们可以一次从结构体或这元组中提取出多个字段存入局部变量。并且和正则表达式类似,它们很简洁,通常只用单行代码就能完成任务。

这一章将以枚举的基础开始,展示数据怎么被关联到枚举选项以及枚举是怎么存储在内存中的。然后我们会展示Rust的模式和\texttt{match}表达式如何简洁地指定基于枚举、结构体、数组、切片的逻辑。模式也可以包含引用、move和\texttt{if}条件,这让它们的功能更加强大。

\section{枚举}\label{enum}

简单的C风格枚举非常直观:
\begin{minted}{Rust}
    enum Ordering {
        Less,
        Equal,
        Greater,
    }
\end{minted}

这里声明了一个有三个可能的值的\texttt{Ordering}类型,这些值被称为\emph{variant}或者\emph{constructor}:\texttt{Ordering::Less}、\texttt{Ordering::Equal}、\texttt{Ordering::Greater}。这个枚举是标准库的一部分,因此Rust代码可以导入它:
\begin{minted}{Rust}
    use std::cmp::Ordering;

    fn compare(n: i32, m: i32) -> Ordering {
        if n < m {
            Ordering::Less
        } else if n > m {
            Ordering::Greater
        } else {
            Ordering::Equal
        }
    }
\end{minted}

或者它的所有constructor:
\begin{minted}{Rust}
    use std::cmp::Ordering::{self, *};  // `*`意思是导入所有的子item

    fn compare(n: i32, m: i32) -> Ordering {
        if n < m {
            Less
        } else if n > m {
            Greater
        } else {
            Equal
        }
    }
\end{minted}

导入constructor之后,我们可以写\texttt{Less}来代替\texttt{Ordering::Less}等,但因为这样不够明显,因此一般认为\emph{不要}导入它们式更好的风格,除非它能是你的代码的可读性更强。

为了导入一个在当前模块中声明的枚举的constructor,可以使用\texttt{self}:
\begin{minted}{Rust}
    enum Pet {
        Orca,
        Giraffe,
        ...
    }

    use self::Pet::*;
\end{minted}

在内存中,C风格的枚举值被存储为整数。有时告诉Rust使用哪些整数会很有用:
\begin{minted}{Rust}
    enum HttpStatus {
        Ok = 200,
        NotModified = 304,
        NotFound = 404,
        ...
    }
\end{minted}

否则,Rust会从0开始自动分配值。

默认情况下,Rust用能容纳所有值的最小的内建整数类型来存储C风格枚举。大多数情况下都是一个单独的字节:
\begin{minted}{Rust}
    use std::mem::size_of;
    assert_eq!(size_of::<Ordering>(), 1);
    assert_eq!(size_of::<HttpStatus>(), 2); // 404不能存储在u8中
\end{minted}

你可以通过添加\texttt{\#[repr]}属性来覆盖Rust选择的内存表示方式。更多的细节见“\nameref{repr}”。

将C风格的枚举转换为整数是允许的:
\begin{minted}{Rust}
    assert_eq!(HttpStatus::Ok as i32, 200);
\end{minted}

然而,反过来把整数转换为枚举是不允许的。和C和C++不同,Rust保证枚举的值只能是\texttt{enum}生命中列出的值之一。未经检查的从整数类型到枚举类型的转换会打破这种保证,所以它是不允许的。你可以写出你自己的带检查的版本:
\begin{minted}{Rust}
    fn http_status_from_u32(n: u32) -> Option<HttpStatus> {
        match n {
            200 => Some(HttpStatus::Ok),
            304 => Some(HttpStatus::NotModified),
            404 => Some(HttpStatus::NotFound),
            ...
            _ => None,
        }
    }
\end{minted}

或者使用\href{https://crates.io/crates/enum_primitive}{\texttt{enum\_primitive}} crate。它包含一个宏可以为你自动生成这种类型的转换代码。

和结构体一样,编译器也可以为你自动生成类似\texttt{==}运算符这样的特性,但你需要显式地要求这样:
\begin{minted}{Rust}
    #[derive(Copy, Clone, Debug, PartialEq, Eq)]
    enum TimeUnit {
        Seconds, Minutes, Hours, Days, Months, Years,
    }
\end{minted}

枚举也和结构体一样可以拥有方法:
\begin{minted}{Rust}
    impl TimeUnit {
        /// 返回该时间单位的复数名词。
        fn plural(self) -> &'static str {
            match self {
                TimeUnit::Seconds => "seconds",
                TimeUnit::Minutes => "minutes",
                TimeUnit::Hours => "hours",
                TimeUnit::Days => "days",
                TimeUnit::Months => "months",
                TimeUnit::Years => "years",
            }
        }

        /// 返回该时间单位的单数名词。
        fn singular(self) -> &'static str {
            self.plural().trim_end_matches('s')
        }
    }
\end{minted}

C风格的枚举就这么多内容了。Rust中最有趣的一类枚举是那些带有数据的枚举。我们将展示这些枚举如何存储在内存中、如何通过添加类型参数将它们变为泛型的,以及如何通过枚举构建复杂的数据结构。

\subsection{带有数据的枚举}

一些程序总是需要显示完整的日期和时间,并且精确到毫秒。但对于大多数程序,显示大概的时间范围会更加友好,例如“两个月以前”。我们可以用之前定义的枚举编写一个新的枚举来实现这一点:
\begin{minted}{Rust}
    /// 一个故意舍入的时间戳,因此我们的程序会显示“6个月以前”
    /// 而不是“February 9, 2016, at 9:49 AM”。
    #[derive(Copy, Clone, Debug, PartialEq)]
    enum RoughTime {
        InThePast(TimeUnit, u32),
        JustNow,
        InTheFuture(TimeUnit, u32),
    }
\end{minted}

这个枚举中的两个variant,即\texttt{InThePast}和\texttt{InTheFuture}都有参数。这些被称为\emph{tuple variant}。就像类元组结构体一样,它们的constructor是创建新的\texttt{RoughTime}值的函数:
\begin{minted}{Rust}
    let four_score_and_seven_years_ago =
        RoughTime::InThePast(TimeUnit::Years, 4 * 20 + 7);

    let three_hours_from_now =
        RoughTime::InTheFuture(TimeUnit::Hours, 3);
\end{minted}

枚举也可以有\emph{struct variant},它们和普通的结构体一样拥有命名字段:
\begin{minted}{Rust}
    enum Shape {
        Sphere { center: Point3d, radius: f32 },
        Cuboid { corner1: Point3d, corner2: Point3d },
    }

    let unit_sphere = Shape::Sphere {
        center: ORIGIN,
        radius: 1.0,
    };
\end{minted}

总的来说,Rust有三种枚举variant,分别对应我们在上一章中展示的三种结构体。没有数据的variant对应类单元结构体。元组variant对应类元组结构体。结构体variant对应有花括号和命名字段的结构体。一个枚举可以同时有这三种variant:
\begin{minted}{Rust}
    enum RelationshipStatus {
        Single,
        InARelationship,
        ItsComplicated(Option<String>),
        ItsExtremelyComplicated {
            car: DifferentialEquation, 
            cdr: EarlyModernistPoem,
        },
    }
\end{minted}

所有种类的constructor都和枚举自身有相同的可见性。

\subsection{内存中的枚举}




