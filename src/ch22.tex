\chapter{unsafe代码}\label{ch22}

\emph{Let no one think of me that I am humble or weak or passive; \\
Let them understand I am of a different kind: dangerous to my enemies, loyal to my friends. \\
To such a life glory belongs.}

系统级编程的乐趣在于,在每一个安全的语言和精心设计的抽象之下,都是不安全的机器语言和比特位。你也可以在Rust中写出这样的代码。

到目前为止本书中介绍的语言部分,例如类型、生命周期、约束检查等都可以自动保证你的程序完全没有内存错误和数据竞争。但这种自动的技术有它的局限性;Rust并不能识别出来很多有价值的技术是安全的。

\emph{unsafe代码}让你可以告诉Rust,“我要使用一些你不能保证安全的特性”。把一个块或者函数标记为unsafe之后,你就可以调用标准库中的\texttt{unsafe}函数、解引用unsafe指针、调用其他语言例如C和C++编写的函数等。Rust的其他安全性检查依然生效:类型检查、生命周期检查、约束检查等仍然和之前一样。unsafe代码只是允许了一小部分额外的特性。

正是因为有了允许超出safe Rust界限的能力,Rust才能实现它自身的大部分基础特性,和C/C++一样,Rust也被用来实现它自己的标准库。unsafe代码可以让\texttt{Vec}类型更高效地管理它的缓冲区;让\texttt{std::io}模块和操作系统交互;让\texttt{std::thread}和\texttt{std::sync}模块提供并发原语。

本章介绍了unsafe特性的一些基础:

\begin{enumerate}
    \item Rust的\texttt{unsafe}块区分开了普通的safe Rust代码和使用unsafe特性的代码。
    \item 你可以把函数标记为\texttt{unsafe},提醒调用者他们必须遵守的一些额外约束来避免未定义行为。
    \item 原始指针和它们的方法运行对内存进行不受限的访问,并允许你构建Rust的类型系统可能禁止的数据结构。Rust的引用虽然安全,但却是受限制的,原始指针正如每个C或C++程序员所知,是一个强大而锋利的工具。
    \item 理解未定义行为的定义将会帮助你理解为什么它们比得到错误结果还要糟糕的多。
    \item unsafe trait,类似于\texttt{unsafe}函数,隐含了每个实现(而不是每个调用者)都要遵守的规则。
\end{enumerate}

\section{unsafe从何而来?}
在本书的开头处,我们曾经展示过一个C程序,它以一种非常令人惊讶的方式崩溃。这是因为它违背了C标准的一个规则。你可以在Rust中实现相同的效果:

\begin{minted}{bash}
    $cat crash.rs
    fn main() {
        let mut a: usize = 0;
        let ptr = &mut a as *mut usize;
        unsafe {
            *ptr.offset(3) = 0x7ffff72f484c;
        }
    }
    $ cargo build
       Compiling unsafe-samples v0.1.0
        Finished debug [unoptimized + debuginfo] target(s) in 0.44s
    $ ../../target/debug/crash
    crash: Error: .netrc file is readable by others.
    crash: Remove password or make file unreadable by others.
    Segmentation fault (core dumped)
    $
\end{minted}

这个程序借用了局部变量\texttt{a}的一个可变引用,然后把它转换成了\texttt{*mut usize}类型的原始指针,然后使用了它的\texttt{offset}方法来产生一个指向三个字之后的位置的指针。这里恰巧存储了\texttt{main}的返回地址。这个程序用一个常量覆盖了返回地址,因此从\texttt{main}返回之后程序的行为就很奇怪。这个崩溃之所以可行,是因为程序错误使用了unsafe的特性——在这个例子中,就是解引用原始指针的能力。

一个unsafe特性通常隐含了一份\emph{合约(contract)}:即一组Rust不能自动强制,但你必须遵守才能避免\emph{未定义行为}的规则。

一份合约超出了普通的类型检查和生命周期检查,它们隐含着一些unsafe特性特定的规则。通常来说,Rust本身完全不知道这些合约,它们只在unsafe特性的文档里得到解释。例如,原始指针类型的合约是禁止解引用一个指向的位置超出原来指向物的末尾的原始指针。这个例子中的表达式\texttt{*ptr.offset(3) = ...}打破了这个合约。但是,正如上面所示,Rust没有任何警告,成功编译了这个程序:它的安全检查并没有检测出来这个违规行为。当你使用unsafe特性时,你作为程序员,需要负责检查你的代码遵守了它们的合约。

很多特性如果想正确使用,都要遵守一定的规则。但这些规则并不是我们这里说的合约,除非它们可能会导致未定义行为。未定义行为是一种Rust假设你的代码中绝对不会出现的行为。例如,Rust假设你不会用别的值覆盖函数的返回地址。通过了Rust通常的安全检查并且遵守了使用到的unsafe特性的合约的代码不可能会出现这样的行为。因为这个程序违反了原始指针的合约,因此它的行为变得未定义,并最终崩溃。

如果你的代码出现了未定义行为,说明你打破了你负责的一部分,Rust拒绝预测结果。从系统库的深处抛出来一个错误并崩溃是一种可能的结果;把你的计算机的控制权交给攻击者是另一种可能的结果。不同的Rust版本可能也会有不同的行为。然而,有时未定义行为不一定会产生可见的结果。例如,如果这里的\texttt{main}函数永远不会返回(可能调用了\texttt{std::process::exit})来提前终止程序),那么错误的返回地址也无关紧要。

你只能在\texttt{unsafe}块或者\texttt{unsafe}函数里使用unsafe特性;我们将在接下来的小节介绍它们。它们让unsafe特性不容易被忽略:通过强迫你写一个\texttt{unsafe}块或者函数,Rust能确保你知道你的代码可能要遵守一些额外的规则。

\section{unsafe块}




\section{原始指针}\label{rawp}

