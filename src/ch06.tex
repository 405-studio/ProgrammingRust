\chapter{表达式}\label{ch06}

\emph{LISP programmers know the value of everything, but the cost of nothing}

\begin{flushright}
    ——Alan Perlis, epigram \#55
\end{flushright}

在这一章中,我们将介绍Rust的\emph{表达式},它是构成Rust函数体和大部分Rust代码的构建块。Rust中大部分都是表达式。在这一章中,我们将探索表达式的力量以及如何克服它的局限。我们还将介绍控制流,它在Rust中完全是以表达式为基础的,最后还要介绍Rust中的基本运算符如何单独和组合工作。

还有一些从技术角度应该划入这一类的概念,例如闭包和迭代器,因为足够重要因此我们之后会用单独的章节介绍它们。现在,我们希望能用尽可能少的页数介绍尽可能多的语法。

\section{表达式语言}

Rust表面上看上去像C家族的语言,但这其实是一个误解。在C语言中,\emph{表达式}和\emph{语句}之间有很大的不同。表达式是一些像这样的代码:
\begin{minted}{C}
    5 * (fahr-32) / 9
\end{minted}
而语句则是像这样的:
\begin{minted}{C}
    for (; begin != end; ++begin) {
        if (*begin == target)
            break;
    }
\end{minted}
表达式有值,但语句没有。

Rust是一种\emph{表达式语言}。这意味着它遵循了起源于Lisp的传统,也就是表达式负责完成所有工作。

在C中,\texttt{if}和\texttt{switch}是语句。它们并不产生值,也不能被用在表达式中间。在Rust中,\texttt{if}和\texttt{match}\emph{可以}产生值。我们已经在\hyperref[ch02]{第2章}中看到过一个产生数字值的\texttt{match}表达式:
\begin{minted}{Rust}
    pixels[r * bounds.0 + c] =
        match escapes(Complex { re: point.0, im: point.1 }, 255) {
            None => 0,
            Some(count) => 255 - count as u8
        };
\end{minted}

一个\texttt{if}表达式可以用于初始化一个变量:
\begin{minted}{Rust}
    let status =
        if cpu.temperature <= MAX_TEMP {
            HttpStatus::Ok
        } else {
            HttpStatus::ServerError  // server melted
        };
\end{minted}

一个\texttt{match}表达式可以被用作函数参数或宏的参数:
\begin{minted}{Rust}
    println!("Inside the vat, you see {}.",
        match vat.contents {
            Some(brain) => brain.desc(),
            None => "nothing of interest"
        });
\end{minted}

这解释了Rust为什么没有C的三元运算符\texttt{(expr1 ? expr2 : expr3))}。在C中,它是一种类似\texttt{if}语句的表达式。在Rust中这种写法是多余的,因为\texttt{if}表达式可以同时实现这两种功能。

C中的大部分控制流工具都是语句,在Rust中则全是表达式。

\section{优先级和结合性}

\hyperref[t6-1]{表6-1}总结了Rust的表达式语法。我们将在这一章中介绍所有这些表达式。运算符按照优先级从高到低的顺序列出。(类似于大多数编程语言,Rust使用\emph{运算符优先级}来决定当表达式中含有多个运算符时的运算顺序。例如,在表达式\texttt{limit < 2 * broom.size + 1}中,\texttt{.}运算符优先级最高,因此会先访问字段。)

\begin{longtable}{p{0.25\textwidth}p{0.35\textwidth}p{0.3\textwidth}}
    \caption{表达式}
    \label{t6-1} \\
    \hline
    \textbf{表达式类型} & \textbf{示例} & \textbf{相关trait} \\
    \hline
    数组字面量      & \texttt{[1, 2, 3]}         & \\
    重复数组字面量  & \texttt{[0; 50]}           & \\
    元组            & \texttt{(6, "crullers")}  & \\
    \cline{1-2}
    组合            & \texttt{(2 + 2)}             & \\
    块              & \texttt{\{ f(); g() \}}      & \\
    控制流表达式     & \texttt{if ok \{ f() \}}     & \\
                    & \texttt{if ok \{ 1 \} else \{ 0 \}}                   & \\
                    & \texttt{if let Some(x) = f() \{ x \} else \{ 0 \}}    & \\
                    & \texttt{match x \{ None => 0, \_ => 1 \}}             & \\
                    & \texttt{for v in e \{ f(v); \}}                       & \texttt{\hyperref[iter]{std::iter::IntoIterator}} \\
                    & \texttt{while ok \{ ok = f(); \}}                     & \\
                    & \texttt{while let Some(x) = it.next() \{ f(x); \}}    & \\
                    & \texttt{loop \{ next\_event(); \}}                    & \\
                    & \texttt{break}                  & \\
                    & \texttt{continue}               & \\
                    & \texttt{return 0}               & \\
    宏调用          & \texttt{println!("ok")}         & \\
    路径            & \texttt{std::f64::consts::PI}   & \\
    \cline{1-2}
    结构体字面量     & \texttt{Point \{x: 0, y: 0\}}     & \\
    \cline{1-2}
    元组字段访问    & \texttt{pair.0}   & \texttt{\hyperref[deref]{Deref}, \hyperref[deref]{DerefMut}} \\
    结构体字段访问  & \texttt{point.x}  & \texttt{\hyperref[deref]{Deref}, \hyperref[deref]{DerefMut}} \\
    方法调用       & \texttt{point.translate(50, 50)} & \texttt{\hyperref[deref]{Deref}, \hyperref[deref]{DerefMut}} \\
    函数调用       & \texttt{stdin()}   & \texttt{\hyperref[fn]{Fn(Arg0, ...) -> T}, \hyperref[fn]{FnMut(Arg0, ...) -> T}, \hyperref[fn]{FnOnce(Arg0, ...) -> T}} \\
    索引            & \texttt{arr[0]}   & \texttt{\hyperref[index]{Index}, \hyperref[index]{IndexMut}, \hyperref[deref]{Deref}, \hyperref[deref]{DerefMut}} \\
    \cline{1-2}
    错误检查        & \texttt{create\_dir("tmp")?}   & \\
    逻辑/位 NOT     & \texttt{!ok}  & \texttt{\hyperref[unop]{Not}} \\
    负             & \texttt{-num}  & \texttt{\hyperref[unop]{Neg}} \\
    解引用          & \texttt{*ptr} & \texttt{\hyperref[deref]{Deref}, \hyperref[deref]{DerefMut}} \\
    借用            & \texttt{\&val}    & \\
    \cline{1-2}
    类型转换    & \texttt{x as u32} \\
    \cline{1-2}
    乘法        & \texttt{n * 2}    & \texttt{\hyperref[biop]{Mul}} \\
    除法        & \texttt{n / 2}    & \texttt{\hyperref[biop]{Div}} \\
    余数(取模) & \texttt{n \% 2}   & \texttt{\hyperref[biop]{Rem}} \\
    \cline{1-2}
    加法        & \texttt{n + 1}    & \texttt{\hyperref[biop]{Add}} \\
    减法        & \texttt{n - 1}    & \texttt{\hyperref[biop]{Sub}} \\
    \cline{1-2}
    左移        & \texttt{n << 1}   & \texttt{\hyperref[biop]{Shl}} \\
    右移        & \texttt{n >> 1}   & \texttt{\hyperref[biop]{Shr}} \\
    \cline{1-2}
    位与        & \texttt{n \& 1}   & \texttt{\hyperref[biop]{BitAnd}} \\
    \cline{1-2}
    位异或      & \texttt{n \^ 1}   & \texttt{\hyperref[biop]{BitXor}} \\
    \cline{1-2}
    位或        & \texttt{n | 1}    & \texttt{\hyperref[biop]{BitOr}}  \\
    \cline{1-2}
    小于        & \texttt{n < 1}    & \texttt{\hyperref[cmp]{std::cmp::PartialOrd}} \\
    小于等于    & \texttt{n <= 1}   & \texttt{\hyperref[cmp]{std::cmp::PartialOrd}} \\
    大于        & \texttt{n > 1}    & \texttt{\hyperref[cmp]{std::cmp::PartialOrd}} \\
    大于等于    & \texttt{n >= 1}   & \texttt{\hyperref[cmp]{std::cmp::PartialOrd}} \\
    等于        & \texttt{n == 1}   & \texttt{\hyperref[equal]{std::cmp::PartialEq}} \\
    不等于      & \texttt{n != 1}   & \texttt{\hyperref[equal]{std::cmp::PartialEq}} \\   
    \cline{1-2}
    逻辑与      & \texttt{x.ok \&\& y.ok}       & \\
    \cline{1-2}
    逻辑或      & \texttt{x.ok || backup.ok}    & \\
    \cline{1-2}
    左闭右开区间 & \texttt{start .. stop}   & \\
    左闭右闭区间 & \texttt{start ..= stop}  & \\
    \cline{1-2}
    赋值        & \texttt{x = val}  & \\
    复合赋值    & \texttt{x *= 1}   & \texttt{\hyperref[assign]{MulAssign}} \\
                & \texttt{x /= 1}   & \texttt{\hyperref[assign]{DivAssign}} \\
                & \texttt{x \%= 1}  & \texttt{\hyperref[assign]{RemAssign}} \\
                & \texttt{x += 1}   & \texttt{\hyperref[assign]{AddAssign}} \\
                & \texttt{x -= 1}   & \texttt{\hyperref[assign]{SubAssign}} \\
                & \texttt{x <<= 1}  & \texttt{\hyperref[assign]{ShlAssign}} \\
                & \texttt{x >>= 1}  & \texttt{\hyperref[assign]{ShrAssign}} \\
                & \texttt{x \&= 1}  & \texttt{\hyperref[assign]{BitAndAssign}} \\
                & \texttt{x \^= 1}  & \texttt{\hyperref[assign]{BitXorAssign}} \\
                & \texttt{x |= 1}   & \texttt{\hyperref[assign]{BitOrAssign}} \\
    \cline{1-2}
    闭包        & \texttt{|x, y| x + y} & \\
\end{longtable}

所有可以链式使用的运算符都是左结合的。也就是说,一条运算链例如\texttt{a - b - c}被组合为\texttt{(a - b) - c},而不是\texttt{a - (b - c)}。这些运算符可以被任意组合:
\begin{minted}{text}
    * / % + - << >> & ^ | && || as
\end{minted}
比较运算符、赋值运算符、范围运算符\texttt{.. }和\texttt{..=}不能被链式使用。

\section{块和分号}

\label{cast}
