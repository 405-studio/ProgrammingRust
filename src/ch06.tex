\chapter{表达式}\label{ch06}

\emph{LISP programmers know the value of everything, but the cost of nothing}

\begin{flushright}
    ——Alan Perlis, epigram #55
\end{flushright}

在这一章中,我们将介绍Rust的\emph{表达式},它是构成Rust函数体和大部分Rust代码的构建块。Rust中大部分都是表达式。在这一章中,我们将探索表达式的力量以及如何克服它的局限。我们还将介绍控制流,它在Rust中完全是以表达式为基础的,最后还要介绍Rust中的基本运算符如何单独和组合工作。

还有一些从技术角度应该划入这一类的概念,例如闭包和迭代器,因为足够重要因此我们之后会用单独的章节介绍它们。现在,我们希望能用尽可能少的页数介绍尽可能多的语法。

\section{表达式语言}

Rust表面上看上去像C家族的语言,但这其实是一个误解。在C语言中,\emph{表达式}和\emph{语句}之间有很大的不同。表达式是一些像这样的代码:
\begin{minted}{C}
    5 * (fahr-32) / 9
\end{minted}
而语句则是像这样的:
\begin{minted}{C}
    for (; begin != end; ++begin) {
        if (*begin == target)
            break;
    }
\end{minted}
表达式有值,但语句没有。

Rust是一种\emph{表达式语言}。这意味着它遵循了起源于Lisp的传统,也就是表达式负责完成所有工作。

在C中,\texttt{if}和\texttt{switch}是语句。它们并不产生值,也不能被用在表达式中间。在Rust中,\texttt{if}和\texttt{match}\emph{可以}产生值。我们已经在\hyperref[ch02]{第2章}中看到过一个产生数字值的\texttt{match}表达式:
\begin{minted}{Rust}
    pixels[r * bounds.0 + c] =
        match escapes(Complex { re: point.0, im: point.1 }, 255) {
            None => 0,
            Some(count) => 255 - count as u8
        };
\end{minted}

一个\texttt{if}表达式可以用于初始化一个变量:
\begin{minted}{Rust}
    let status =
        if cpu.temperature <= MAX_TEMP {
            HttpStatus::Ok
        } else {
            HttpStatus::ServerError  // server melted
        };
\end{minted}

一个\texttt{match}表达式可以被用作函数参数或宏的参数:
\begin{minted}{Rust}
    println!("Inside the vat, you see {}.",
        match vat.contents {
            Some(brain) => brain.desc(),
            None => "nothing of interest"
        });
\end{minted}

这解释了Rust为什么没有C的三元运算符\texttt{(expr1 ? expr2 : expr3))}。在C中,它是一种类似\texttt{if}语句的表达式。在Rust中这种写法是多余的,因为\texttt{if}表达式可以同时实现这两种功能。

C中的大部分控制流工具都是语句,在Rust中则全是表达式。

\section{优先级和关联性}

\label{cast}
